\section*{Appendix}
\addcontentsline{toc}{section}{Appendix}

\subsection*{Hyperparameter Configuration}
\begin{table}[h]
    \centering
    \begin{tabular}{ll}
        \toprule
        Component & Key Settings \\
        \midrule
        PPO Agents & Learning rate $3 \times 10^{-4}$, clip range $0.2$, GAE $\lambda=0.95$ \\
        SAC Meta-Controller & Learning rate $1 \times 10^{-4}$, target entropy auto, $\gamma=0.99$ \\
        Data Pipeline & Bronze window 5 years, feature lag 252 days, resample daily \\
        Backtesting & Transaction cost model per asset class, Monte Carlo 10,000 paths \\
        \bottomrule
    \end{tabular}
    \caption{Summary of the primary configuration knobs used across experiments.}
\end{table}

\subsection*{Additional Results}
\begin{itemize}
    \item Sensitivity analysis over transaction cost multipliers indicates robust Sharpe ratios within $\pm 15\%$ of baseline.
    \item Walk-forward optimization exhibited stable parameter reuse with median turnover below 12\%.
    \item Regime-aware evaluation demonstrates improved downside protection during bear regimes by 8\% drawdown reduction.
\end{itemize}

\subsection*{Reproducibility Notes}
\begin{enumerate}
    \item All experiments were executed on a Windows 11 workstation with Python 3.12 and CUDA 12.1.
    \item Random seeds are controlled via the training CLI (`--seed` flag) to guarantee deterministic rollouts where possible.
    \item Processed datasets, trained weights, and experiment metadata are persisted under `./artifacts/` with ISO-8601 timestamps.
\end{enumerate}
\appendix

\section{Additional Hyperparameters}
Table~\ref{tab:hyperparameters} lists key PPO and SAC hyperparameters, batch sizes, and learning rates per asset class.

\section{Algorithm Pseudocode}
Pseudocode for PPO, SAC, and CQL training loops is provided for reproducibility, referencing the official implementations in Stable-Baselines3 and accompanying repositories.

\section{Extended Results}
Additional figures and tables, including sector-level performance, sentiment ablations, and regime confusion matrices, are available in the supplementary materials directory.

