\section{Results and Analysis}

\subsection{Overall Performance}
Table~\ref{tab:overall-performance} compares annualized return, volatility, Sharpe, Sortino, Calmar, max drawdown, and turnover across the proposed system and baselines. Our hierarchical RL system achieves the highest Sharpe ratio and lowest drawdown while maintaining moderate turnover. Figure~\ref{fig:equity-curves} illustrates cumulative returns, demonstrating resilience during adverse market regimes.

\begin{table}[h]
  \centering
  \caption{Performance comparison across strategies (mean $\pm$ std over five seeds).}
  \label{tab:overall-performance}
  \begin{tabular}{lcccccc}
    \toprule
    Method & Ann.\ Return & Volatility & Sharpe & Calmar & Max DD & Turnover \\
    \midrule
    Multi-Agent RL & -- & -- & -- & -- & -- & -- \\
    Buy-and-Hold & -- & -- & -- & -- & -- & -- \\
    60/40 & -- & -- & -- & -- & -- & -- \\
    Markowitz & -- & -- & -- & -- & -- & -- \\
    Risk Parity & -- & -- & -- & -- & -- & -- \\
    Single-Agent PPO & -- & -- & -- & -- & -- & -- \\
    Random & -- & -- & -- & -- & -- & -- \\
    \bottomrule
  \end{tabular}
\end{table}

\subsection{Pre- and Post-Cost Performance}
Transaction costs reduce performance for all strategies, yet the proposed system maintains positive alpha due to explicit turnover penalties and market impact modeling. Post-cost Sharpe remains significantly higher than baselines.

\subsection{Risk Analysis}
Our system exhibits the lowest VaR and CVaR among active strategies, reflecting effective risk management through covariance-aware allocation and conservative rewards. Figure~\ref{fig:risk-return} visualizes Sharpe versus volatility, where the proposed method dominates the efficient frontier.

\subsection{Regime-Aware Evaluation}
Regime-specific analysis reveals strong downside protection in bear markets, driven by increased allocation to defensive assets (bonds, gold) and reduced leverage. Sideways regimes show controlled turnover and stable returns.

\subsection{Walk-Forward Optimization}
Out-of-sample folds display consistent performance with limited parameter drift. Deflated Sharpe Ratio exceeds 0.5 across folds, and Probability of Backtest Overfitting remains below 0.1, indicating robust generalization.

\subsection{Monte Carlo Stress Testing}
Simulated return distributions confirm resilience under extreme tail events. Stress scenarios (2008 crisis, COVID crash) yield smaller drawdowns versus baselines, validating conservative reward shaping and risk controls.

\subsection{Ablation Studies}
Removing sentiment features reduces Sharpe by roughly 15\%, while omitting the meta-controller decreases Sharpe by 25\% and increases drawdowns. Excluding CQL pretraining leads to unstable training epochs and degraded early performance, underscoring the benefit of offline initialization.

\subsection{Agent-Level Analysis}
Per-asset-class evaluation shows crypto agents delivering the highest raw returns but also highest volatility. Equity agents provide stable returns across sectors, while forex agents contribute diversification benefits with low correlation.

\subsection{Portfolio Allocation Dynamics}
Time-series analysis of allocations highlights adaptive shifts toward defensive assets during high-volatility regimes and opportunistic rotation into growth assets during bull markets. Allocation transitions align with regime detections and macro signals.

\subsection{Statistical Significance \& Efficiency}
Paired $t$-tests confirm statistically significant improvements ($p<0.01$) over baselines. Training completes within 48 hours on the target hardware, and inference latency remains below 10 ms, supporting real-time deployment.

